\documentclass[a4paper]{exam}

\usepackage{geometry}
\usepackage{graphicx}
\usepackage{hyperref}
\usepackage{titling}

\printanswers

\title{Weekly Challenge 01: Comparison}
\author{CS/MATH 113 Discrete Mathematics}
\date{Spring 2024}

\qformat{{\large\bf \thequestion. \thequestiontitle}\hfill}
\boxedpoints

\begin{document}
\maketitle

\begin{questions}
  
\titledquestion{How about them apples?}
  \begin{minipage}{.3\linewidth}
  \centerline{\includegraphics[width=\textwidth]{picard}}
\end{minipage}
\begin{minipage}{.65\linewidth}
  The \href{https://en.wikipedia.org/wiki/Replicator_(Star_Trek)}{replicator} aboard USS Enterprise has developed a fault---synthesized apples have insufficient nutrition but are otherwise identical to regular apples. Doctor \href{https://memory-alpha.fandom.com/wiki/Beverly_Crusher}{Beverly Crusher} is on the case. Scanning a bunch of apples, her \href{https://en.wikipedia.org/wiki/Medical_tricorder}{tricorder} can indicate if the bunch contains any faulty apples, but it cannot identify them.
\end{minipage}
\begin{parts}
  \part Dr. Crusher is investigating a bunch of 5 apples out of which 1 is known to be faulty. Describe how she can identify the faulty apple in no more than 3 tricorder scans.
  \part What is the minimum number of scans that Dr. Crusher needs to perform in order to guarantee finding the single faulty apple in a bunch of size $n$? Justify your answer.
\end{parts}

\begin{solution}
\begin{parts}
        \part Dr. Crusher should first divide the apple in two groups of two and one will be left without a pair. She should first scan the first pair of apples, if there is a faulty one among the pair, she can scan the pair again, this will be her second scan, among the first pair which includes the faulty apple, after scanning the first apple from the first pair the tricorder will indicate if that apple is faulty in the case that the apple is not faulty, it is safe to assume that the other apple in the pair is the faulty one. To make sure her findings are correct she can scan the apple and use her third attempt. In case the first pair has no faulty apple, she can repeat the same steps and find out the faulty apple from the second pair which will require her to begin with her second attempt, and then use her third attempt to identify the faulty apple among the second pair. In case there is no faulty apple in the second pair, the last apple will of course be the faulty one. She can also use her last attempt on the third apple to verify the case.
        \part If n is over 2 then: If n is odd, then she needs atleast (n+1)/2 scans. If n is even (n/2)+1 scans are needed. Suppose that n is 5 which is an odd number, by applying the formula the minimum number of scans needed would be 3 which is also the case in part a.
\end{parts}
\end{solution}

\end{questions}

\end{document}

%%% Local Variables:
%%% mode: latex
%%% TeX-master: t
%%% End:
